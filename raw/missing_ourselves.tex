
\begin{verse}
it struck me today that maybe we are afraid to die \\
because we will miss our bodies \\
those awkward containers of our lives \\
that bag of sores through which \\
we hold our babies and lovers \\
family, friends, strangers and \\
the occasional patient listener \\
who allows us to settle into ourselves \\
and relax, feel the lightness of our true selves, \\
understand that we are just connections \\
embodied in this funny little form that \\
moves and sings and dances and cries \\
and laughs and runs and falls \\
over and over, picking up to try again \\
never ceasing our work until one day \\
we loose it, have nothing to move \\
no way to say everything we should have, \\
to reach out and hold you in your grief, \\
to cry beside you and stroke your hair once more.

nothing that is permenant could ever be precious to us \\
because we could always return to it, take it for granted. \\
but not our bodies, or lives or our time together \\
because they never cease standing up and falling down \\
working, resting, finding, losing, holding and releasing \\
and always working to hold back the dam of change though \\
each moment we are hurled forward into the unknown

we are taught to fear death because of the unknown of what's next\\
but the unknown is our most quotidian sensation\\
because we know perfectly well what happens when we die.\\
we know we won't hold our children,\\
roll out of bed and place our feet on the cold floor,\\
float beneath a wave as it rolls over us,\\
nestle into our lover's hair as we slowly wake.

and each moment we know that we will miss this sadness, \\
this effort, the long road we walk between then and now, \\
into whatever is next with the ability to change the outcome \\
not merely a passive observer. each moment is an offer to \\
change, another alternative and chance to step into a blessing. \\
who wouldn't want to keep taking that chance? 
\end{verse}
